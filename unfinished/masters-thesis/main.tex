\documentclass[oneside,final,14pt]{extreport}

\usepackage[T2A]{fontenc}
\usepackage[utf8]{inputenc}
\usepackage[russian]{babel}

\usepackage{vmargin}
\setpapersize{A4}
\setmarginsrb{2cm}{1.5cm}{1cm}{1.5cm}{0pt}{0mm}{0pt}{13mm}
\usepackage{indentfirst}
\sloppy

\setcounter{tocdepth}{5}

\begin{document}

\input{target/CURRENT_VERSION}

\tableofcontents

\begin{abstract}
  Это вводный абзац в начале документа.
\end{abstract}

\part{Часть 1}

ААА

\chapter{Глава 11}

АААААААА

\section{Раздел 111}

FF

\subsection{Подраздел 1}

ПДРЗД

\subsubsection{Под-подраздел 1}

Да

\paragraph{Тест 11}

ааа

\subparagraph{Test 22}

Сложение коммутативно: $a + b = b + a$

\section{Предисловие}
 Этот текст будет на русском языке. Это демонстрация того, что символы кириллицы
 в сгенерированном документе (Compile to PDF) отображаются правильно.
 Для этого Вы должны установить нужный  язык (russian) 
и необходимую кодировку шрифта (T2A).
 
\section{Математические формулы}
Кириллические символы также могут быть использованы в математическом режиме.
 
\begin{equation}
  S_\textup{ис} = S_{123}
\end{equation}

\begin{thebibliography}{00}

\bibitem{knuth:tex} Д. Кнут
\emph{Всё про TeX} Издательство ААА
Протвино, 1983

\end{thebibliography}
\end{document}
